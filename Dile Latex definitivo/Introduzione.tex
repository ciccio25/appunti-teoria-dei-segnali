\section*{Introduzione}

 

Appunti ordinati, con approfondimenti passo-passo, del corso di Teoria dei segnali (ora Segnali Determinati e aleatori) per il corso di laurea in Ingegneria Elettronica 
presso l’Università Politecnica delle Marche. \newline
        



Le fonti degli appunti sono le seguenti: 

\begin{itemize}
    
    \item Slide del corso del prof Franco Chiaraluce Segnali Determinati E Aleatori A.A. 2024/2025
    

\end{itemize}

\begin{tcolorbox}
    Negli appunti ci saranno delle piccole appendici dentro a questi mini-paragrafi  
    su come si leggono in italiano le varie formule matematiche,  
    e lascerò link su possibili approfondimenti matematici usando le animazioni, 
    così è più facile comprendere la materia.
\end{tcolorbox}


Per le lettere greche che ci saranno nel corso, vi consiglio di visitare questo sito \\ 
\url{https://www.rapidtables.org/it/math/symbols/greek_alphabet.html} \break 
in cui è disponibile anche la pronuncia vocale delle lettere. \newline

Le figure senza l'indicazione sono presi dalle dispense del corso, 
sennò è esplicitato il link dell'immagine. \newline 

Gli appunti del corso scritti dal prof sono sufficienti per passare il corso e mi congratulo pubblicamente con il prof per essere un prof con la P maiuscola, ce ne sono veramente pochi 
che ti trasmettono cosa significa la passione dell'insegnamento.  \newline 

Inoltre, dopo il ricevimento che ho avuto con il professore, per l'orale è necessario capire graficamente il concetto; 
le dimostrazioni matematiche servono, ma non sono utili per l'esame, basta solo capire, se ci vengono mostrati, 
perchè si fanno quei passaggi e/o semplificazioni matematici. \newline 

Questi appunti scritti dal sottoscritto sono degli approfondimenti in più, una sistemazione 
grafica diversa rispetto al prof e un riassunto. \newline 

Per qualsiasi domanda, scrivimi a \href{mailto:rossini.stefano.appunti@gmail.com}{rossini.stefano.appunti@gmail.com} \newline

Buono studio e buona lettura \newline

\newpage 





